\documentclass[16pt]{article}
% Codificación y márgenes
\usepackage[utf8]{inputenc}
\usepackage[margin=1in]{geometry}
\usepackage{fancyhdr}
\usepackage{amsmath, amssymb}
% Fuentes y espaciado
\usepackage{setspace}
\doublespacing
\usepackage{times}
\usepackage{parskip}
%\setlength{\parindent}{1.27cm}
% Encabezado APA
\pagestyle{fancy}
\fancyhead[L]{\shorttitle}
\fancyhead[R]{\thepage}
\begin{document}
\newcommand{\shorttitle}{Quackternion: Posicion y Orientacion de Objetos en un espacio 3D}\begin{document}\section*{Definicion de Object} 
Definimos un objeto en el espacio tridimensional como el par
\[ \text{Object} = \big( \vec{p}, \, q \big) \]
donde:
\[
\vec{p} = \begin{bmatrix} x \\ y \\ z \end{bmatrix} \in \mathbb{R}^3
\quad \text{y} \quad
q = \begin{bmatrix} w \\ x \\ y \\ z \end{bmatrix} \in \mathbb{H}.
\] 

$p_{i}$: Posición inicial de Objeto, antes de aplicar la transformacion 

$q_{i}$: Orientacion inicial del Objeto, antes de aplicar la transformacion 

$\Delta \vec{p}_{local}$: Vector de Traslacion relativo al objeto

$\Delta \vec{p}_{global}$: Vector de Traslacion absoluto

$q_{rot}$: Cuaternión de rotación aplicado 

$p_{i+1}$: Posición final del Objeto, despues de aplicar la transformacion 

$q_{i+1}$: Orientación final del Objeto, despues de aplicar la transformacion 

\subsubsection*{Formulas para la Traslacion}
\[ (0, \Delta \vec{p}_{local}) = v = \begin{bmatrix} 0 \\ d_x \\ d_y \\ d_z \end{bmatrix}, \quad\Delta \vec{p}_{\text{global}} = q \, (0, \Delta \vec{p}_{local}) \, q^{-1} = q \, v \, q^{-1} = \begin{bmatrix} q_w \\ q_x \\ q_y \\ q_z \end{bmatrix} \cdot \begin{bmatrix} 0 \\ d_x \\ d_y \\ d_z \end{bmatrix} \cdot \begin{bmatrix} q_w \\ -q_x \\ -q_y \\ -q_z \end{bmatrix} \]

\[ \vec{p}_{i+1} = \vec{p}_{i} + \Delta \vec{p}_{global}\]

\subsubsection*{Formulas para la Rotacion}
\[ q_{rot}(\theta, axis) = 
\begin{cases}
\; (\cos(\tfrac{\theta}{2}),\;\sin(\tfrac{\theta}{2}),\;0,\;0) & \text{si } X\\
\; (\cos(\tfrac{\theta}{2}),\;0,\;\sin(\tfrac{\theta}{2}),\;0) & \text{si } Y\\
\; (\cos(\tfrac{\theta}{2}),\;0,\;0,\;\sin(\tfrac{\theta}{2})) & \text{si } Z
\end{cases}
\]

\[ q_{i+1} = q_{rot} \cdot q_{i}\]

\subsubsection*{Producto de Cuaterniones}
Sean dos cuaterniones definidos como:
\[
q_1 = \begin{bmatrix} w_1 \\ x_1 \\ y_1 \\ z_1 \end{bmatrix}, \quad q_2 = \begin{bmatrix} w_2 \\ x_2 \\ y_2 \\ z_2 \end{bmatrix}
\]
El producto de cuaterniones $q = q_1 \cdot q_2$ se define como:
\[
q = \begin{bmatrix} w \\ x \\ y \\ z \end{bmatrix} = \begin{bmatrix} w_1 w_2 - x_1 x_2 - y_1 y_2 - z_1 z_2 \\ w_1 x_2 + x_1 w_2 + y_1 z_2 - z_1 y_2 \\ w_1 y_2 - x_1 z_2 + y_1 w_2 + z_1 x_2 \\ w_1 z_2 + x_1 y_2 - y_1 x_2 + z_1 w_2 \end{bmatrix}
\]
Aquí $w$ es la parte real y $(x, y, z)$ son las componentes imaginarias.

\subsubsection*{Quaterniones a Ángulos de Euler}
Los ángulos de Euler (roll, pitch, yaw) en radianes, para la orientacion final:
\[
 \text{roll}\,(X) = \phi = \arctan2(2(q_w \cdot q_x + q_y \cdot q_z), 1 - 2(q_x^2 + q_y^2))
 \]
\[
 \text{pitch}\,(Y) = \theta = \arcsin(2(q_w \cdot q_y - q_z \cdot q_x))
 \]
\[
 \text{yaw}\,(Z) = \psi = \arctan2(2(q_w \cdot q_z + q_x \cdot q_y), 1 - 2(q_y^2 + q_z^2))
 \]

\newpage
 \section*{Object 1}
Consideremos un objeto particular, al que llamaremos $Object 1$, definido como:
\[ 
\text{Object}_1 = \big( \vec{p}_1, \, q_1 \big)
\]
donde inicialmente:
\[
\vec{p}_1 = \begin{bmatrix} 10.0 \\ 0.0 \\ 0.0 \end{bmatrix} \in \mathbb{R}^3
\quad \text{y} \quad
q_1 = \begin{bmatrix} 1 \\ 0 \\ 0 \\ 0 \end{bmatrix} \in \mathbb{H}.
\]

\subsection*{Rotacion en el eje Y de $\mathbf{180^\circ}$ }
\subsubsection*{Parámetros iniciales}
\[ q_{1, i} = \begin{bmatrix} 1 \\ 0 \\ 0 \\ 0 \end{bmatrix}, \quad \theta_y = 180^\circ = 3.1416\;  rad  \]

\subsubsection*{Rotación alrededor del eje Y}
\[ q_{rot}(180^\circ, y) = \begin{bmatrix} \cos(\tfrac{\theta}{2}) \\ 0 \\ \sin(\tfrac{\theta}{2}) \\ 0 \end{bmatrix}= \begin{bmatrix} 0 \\ 0 \\ 1 \\ 0 \end{bmatrix}\]

\subsubsection*{Actualización de la orientación}
\[ q_{1, i+1} = q_{rot} \cdot q_{1, i} = = \begin{bmatrix} 0 \\ 0 \\ 1 \\ 0 \end{bmatrix} \cdot \begin{bmatrix} 1 \\ 0 \\ 0 \\ 0 \end{bmatrix}\]
\[ q_{1, i+1} = \begin{bmatrix} 0 \\ 0 \\ 1 \\ 0 \end{bmatrix} \]

\subsection*{Traslacion relativa de (10, 0, 0) }
\subsubsection*{Parametros iniciales}
\[ \vec{p}_{1, i} = \begin{bmatrix} p_x \\ p_y \\ p_z \end{bmatrix} = \begin{bmatrix} 10 \\ 0 \\ 0 \end{bmatrix}, \quad 
\Delta \vec{p}_{1,local} = \begin{bmatrix} d_x \\ d_y \\ d_z \end{bmatrix} = \begin{bmatrix} 10 \\ 0 \\ 0 \end{bmatrix}, \quad 
q_{1, i} = \begin{bmatrix} q_w \\ q_x \\ q_y \\ q_z \end{bmatrix} = \begin{bmatrix} 0 \\ 0 \\ 1 \\ 0 \end{bmatrix} \]

\subsubsection*{Rotación del Vector Local hacia el espacio global}

\[ v = \begin{bmatrix} 0 \\ d_x \\ d_y \\ d_z \end{bmatrix} = \begin{bmatrix} 0 \\ 10 \\ 0 \\ 0 \end{bmatrix}\]\[ \Delta \vec{p}_{\text{global}} = q \cdot v \cdot q^{-1} = \begin{bmatrix} - (q_x d_x + q_y d_y + q_z d_z) \\ q_w d_x + q_y d_z - q_z d_y \\ q_w d_y - q_x d_z + q_z d_x \\ q_w d_z + q_x d_y - q_y d_x \end{bmatrix} \cdot \begin{bmatrix} q_w \\ -q_x \\ -q_y \\ -q_z \end{bmatrix} \]

\[ \Delta \vec{p}_{\text{global}} = \begin{bmatrix} - (q_x d_x + q_y d_y + q_z d_z) \\ q_w d_x + q_y d_z - q_z d_y \\ q_w d_y - q_x d_z + q_z d_x \\ q_w d_z + q_x d_y - q_y d_x \end{bmatrix} \cdot \begin{bmatrix} 0 \\ -0 \\ -1 \\ -0 \end{bmatrix} \]

\[ \Delta \vec{p}_{global} = \begin{bmatrix} -10 \\ 0 \\ 0 \end{bmatrix} 
\]\textbf{Traslacion de la posición global}

\[ \vec{p}_{1, i+1} = \vec{p}_{1, i} + \Delta \vec{p}_{1, global} =  = \begin{bmatrix} 10 + -10 \\ 0 + 0 \\ 0 + 0 \end{bmatrix}\] 

\[\vec{p}_{1,i+1} = \begin{bmatrix} 0 \\ 0 \\ 0 \end{bmatrix}
\]
 \subsection*{Valores Finales}
\[
 \vec{p}_1 = \begin{bmatrix} 0 \\ 0 \\ 0 \end{bmatrix}\quad , \quad
q_1 = \begin{bmatrix} 0 \\ 0 \\ 1 \\ 0 \end{bmatrix}\]
\subsubsection*{Orientacion final como Ángulos de Euler}
Los ángulos de Euler (roll, pitch, yaw), para la orientacion final:
\[
 \text{roll}\,(X) = \phi = \arctan2(2(0 \cdot 0 + 1 \cdot 0), 1 - 2(0^2 + 1^2))
 = 3.1416 \; rad = 180^\circ  \]
\[
 \text{pitch}\,(Y) = \theta = \arcsin(2(0 \cdot 1 - 0 \cdot 0))
 = 0 \; rad = 0^\circ \]
\[
 \text{yaw}\,(Z) = \psi = \arctan2(2(0 \cdot 0 + 0 \cdot 1), 1 - 2(1^2 + 0^2))
 = 3.1416 \; rad = 180^\circ \]
\end{document}
